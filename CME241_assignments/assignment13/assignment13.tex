\documentclass{article}

% Language setting
% Replace `english' with e.g. `spanish' to change the document language
\usepackage[english]{babel}

% Set page size and margins
% Replace `letterpaper' with`a4paper' for UK/EU standard size
\usepackage[letterpaper,top=2cm,bottom=2cm,left=3cm,right=3cm,marginparwidth=1.75cm]{geometry}

% Useful packages
\usepackage{amsmath}
\usepackage{amssymb}
\usepackage{graphicx,float}
\usepackage{xcolor}
\usepackage[colorlinks=true, allcolors=blue]{hyperref}

\title{Assignment 13}
\author{Miao-Chin Yen}

\begin{document}
\maketitle

\section*{Problem 1}
Reference   \href{https://github.com/miaochin/RL-book/tree/master/CME241_assignments/assignment13}{\textcolor{blue}{glie\textunderscore mc\textunderscore tabular.py}} and \href{https://github.com/miaochin/RL-book/tree/master/CME241_assignments/assignment13}{\textcolor{blue}{glie\textunderscore mc\textunderscore func\textunderscore approx.py}}.\\
Note that tabular method can be seen as a kind of linear function approximation.


\section*{Problem 2}
Reference   \href{https://github.com/miaochin/RL-book/tree/master/CME241_assignments/assignment13}{\textcolor{blue}{glie\textunderscore sarsa\textunderscore tabular.py}} and \href{https://github.com/miaochin/RL-book/tree/master/CME241_assignments/assignment13}{\textcolor{blue}{glie\textunderscore sarsa\textunderscore func\textunderscore approx.py}}.\\
Note that tabular method can be seen as a kind of linear function approximation.

\section*{Problem 3}
Reference   \href{https://github.com/miaochin/RL-book/tree/master/CME241_assignments/assignment13}{\textcolor{blue}{q\textunderscore learning\textunderscore tabular.py}} and \href{https://github.com/miaochin/RL-book/tree/master/CME241_assignments/assignment13}{\textcolor{blue}{q\textunderscore learning\textunderscore func\textunderscore approx.py}}.\\
Note that tabular method can be seen as a kind of linear function approximation.





\end{document}