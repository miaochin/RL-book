\documentclass{article}

% Language setting
% Replace `english' with e.g. `spanish' to change the document language
\usepackage[english]{babel}

% Set page size and margins
% Replace `letterpaper' with`a4paper' for UK/EU standard size
\usepackage[letterpaper,top=2cm,bottom=2cm,left=3cm,right=3cm,marginparwidth=1.75cm]{geometry}

% Useful packages
\usepackage{amsmath}
\usepackage{amssymb}
\usepackage{graphicx,float}
\usepackage{xcolor}
\usepackage[colorlinks=true, allcolors=blue]{hyperref}

\title{Assignment 6}
\author{Miao-Chin Yen}

\begin{document}
\maketitle

\section*{Problem 1}
\hspace{1em}Assuem utility function $U(x) = x-\frac{\alpha x^2}{2}$ and $x \sim \mathcal{N}(\mu,\sigma^{2})$, calculate:\\
1. Expected Utility $E[U(x)]$:\\
$$E[U(x)] = E[x-\frac{\alpha x^2}{2}] = E[x]-\frac{\alpha}{2}E[x^2] = \mu-\frac{\alpha}{2}(\sigma^2 + \mu^2)$$
2. Certainty-Equivalent Value: $x_{CE}$
$$ x_{CE} = U^{-1}(E[U(x)]) = U^{-1}( \mu-\frac{\alpha}{2}(\sigma^2\ + \mu^2)) \Rightarrow x_{CE} = \frac{1 \pm\sqrt[]{\alpha^2\mu^2+\alpha^2\sigma^2-2\alpha\mu +1}}{\alpha}
$$
3. Absolute Risk-Premium $\pi_{A}$
$$\pi_{A} =E[x]-x_{CE} = \mu-\frac{1 \pm\sqrt[]{\alpha^2\mu^2+\alpha^2\sigma^2-2\alpha\mu +1}}{\alpha} $$
Invest $z$ dollars in risky asset and $1-z$ dollars in riskless asset. Let $W$ denote the wealth in one year where $W \sim \mathcal{N}(1+r+z(\mu-r),z^2\sigma^{2})$. 
Our goal is to maximize $E[U(W)]$.
$$ \max_{z} E(U(W)) = 1+r+z(\mu-r) -\frac{\alpha}{2}(z^2\sigma^{2} + (1+r+z(\mu-r))^2 )  $$
F.O.C:
$$ \mu-r -\frac{\alpha}{2}(2z^{*}\sigma^2 + 2(\mu-r)(1+r+z^{*}(\mu-r)))=0$$
$$\mu-r -\alpha z^{*} \sigma^2 -\alpha (\mu-r)(1+r)-\alpha z^{*}(\mu-r)^2 = 0$$
$$z^{*}(-\alpha \sigma^2 - \alpha(\mu-r)^2) = (\mu-r)(\alpha + \alpha r-1) $$
$$z^{*}= \frac{(\mu-r)(\alpha + \alpha r-1)}{-\alpha \sigma^2 - \alpha(\mu-r)^2}$$
Let $\mu=0.3,\text{ }r = 0.05,\text{ }\sigma = 0.2$:
\begin{center}
\includegraphics[scale=0.25]{risk_aversion}
\end{center}
We can see that as our risk aversion level increases (we are afraid of risks), we would tend to invest in riskless asset.
\section*{Problem 3}
(a) Write down the two outcomes for wealth $W$ at the end of your single bet of $f \cdot W_{0}$
\begin{itemize}
\item[i.] $W=f\cdot W_0 (1+\alpha) + (1-f)\cdot W_0 = f\cdot W_0 \cdot \alpha + W_0$
\item[ii.] $W=f\cdot W_0 (1-\beta) + (1-f)\cdot W_0 = - f\cdot W_0 \cdot \beta + W_0$
\end{itemize}
(b) Write down the two outcomes for $\log$ (Utility) of $W$.
\begin{itemize}
\item[i.] $log(W) =log(f\cdot W_0 \cdot \alpha + W_0)$
\item[ii.] $log(W)=log(-f\cdot W_0 \cdot \beta + W_0)$
\end{itemize}
(c) Write down $\mathbb{E}[\log (W)]$.
$$\mathbb{E}[\log (W)] = p\cdot log(f\cdot W_0 \cdot \alpha + W_0) + q \cdot log(-f\cdot W_0 \cdot \beta + W_0)$$
(d) Take the derivative of $\mathbb{E}[\log (W)]$ with respect to $f$.\\
$$ p \cdot \frac{W_0 \cdot \alpha}{f\cdot W_0 \cdot \alpha + W_0} + q \cdot \frac{W_0 \cdot \beta}{f\cdot W_0 \cdot \beta - W_0}$$
(e) Set this derivative to 0 to solve for $f^{*}$. Verify that this is indeed a maxima by evaluating the second derivative at $f^{*}$. This formula for $f^{*}$ is known as the Kelly Criterion.\\
$$ p \cdot \frac{W_0 \cdot \alpha}{f^*\cdot W_0 \cdot \alpha + W_0} + q \cdot \frac{W_0 \cdot \beta}{f^*\cdot W_0 \cdot \beta - W_0} = 0$$
$$ f^*\cdot p \cdot\alpha \cdot \beta \cdot W_{0}^{2} - p \cdot  \alpha \cdot W_{0}^{2} = -f^* \cdot q  \cdot \alpha \cdot \beta \cdot W_{0}^{2} - q \cdot \beta \cdot W_{0}^{2} $$
$$f^* = \frac{p \cdot \alpha W_{0}^{2}-q\cdot \beta W_{0}^{2}}{W_{0}^{2}\cdot \alpha \cdot \beta} = \frac{p \cdot \alpha -q\cdot \beta }{\alpha \cdot \beta}$$
Second derivative:
$$-\frac{p\cdot (w_0\cdot\alpha)^2}{(f\cdot W_0 \cdot \alpha+W_0)^2}-\frac{q\cdot (w_0\cdot\beta)^2}{(f\cdot W_0 \cdot \beta-W_0)^2}$$
Since
$$-\frac{p\cdot (w_0\cdot\alpha)^2}{(f^{*}\cdot W_0 \cdot \alpha+W_0)^2}-\frac{q\cdot (w_0\cdot\beta)^2}{(f^{*}\cdot W_0 \cdot \beta-W_0)^2} < 0,$$
this is indeed a maxima.\\
(f)$$ f^* = \frac{p \cdot \alpha -q\cdot \beta }{\alpha \cdot \beta} $$
 Clearly, if $\alpha$ is higher, we would get more back if we win the bet. If we only focus on the numerator, we can see that as $\alpha$ increases, we would bet more. Same idea is applied for $\beta$. For probability $p$, if $p$ is higher, we would have a higher chance to get more money. Hence, we would like to bet more.






\end{document}